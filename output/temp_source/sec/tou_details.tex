\textbf{Temporal Order Understanding (TOU)}

The Temporal Order Understanding task evaluates models' ability to comprehend chronological sequences and identify temporal relationships within multi-image sequences. This task addresses a fundamental aspect of temporal cognition---understanding how events unfold over time and their sequential relationships. Our TOU task is constructed using the Satellite Images Time Series Change Caption (SITSCC) dataset, which contains 1,000 groups of remote sensing image sequences. Each group captures the same geographical location across multiple time points, with each sequence containing an average of 5 or more images, covering various temporal scales from seasonal changes to multi-year urban development. We randomly partition the dataset into 800 groups for training and 200 groups for evaluation.

The TOU task encompasses five complementary subtasks at different granularities and difficulty levels, evaluating distinct aspects of temporal sequence understanding:
\begin{enumerate}
\item \textbf{Temporal Anomaly Localization (TAL)} presents models with a sequence of five images where four images from the same location form a coherent temporal progression, while one anomalous image either exhibits temporal inconsistency or originates from a different location, disrupting the evolutionary logic. Given a sequence where one image is anomalous, the model can readily identify its temporal position through conspicuous visual patterns and inconsistencies in temporal evolution, outputting its temporal index. This task primarily evaluates the model's ability to recognize the positional identification of specific time points within a sequence.

\item \textbf{Image Sequence Reordering (ISR)} presents models with shuffled image sequences and requires them to predict the correct temporal ordering. Given an image sequence with true temporal order, the model receives a shuffled version and must predict an ordering that approximates the ground truth order. This task evaluates the model's ability to understand temporal structure and sequential evolution patterns across time periods.

\item \textbf{Sequential Order Verification (SOV)} assesses whether models can distinguish between temporally consistent and inconsistent image sequences. Given an image sequence, the order may be temporally consistent (e.g., correctly ordered) or shuffled (e.g., randomly ordered). Models are presented with both correctly ordered sequences and shuffled sequences, requiring them to determine whether the sequence represents a reasonable temporal evolution process, outputting True or False. This binary classification task tests the model's sensitivity to temporal inconsistencies.

\item \textbf{Pairwise Order Verification (POV)} is a simplified version of SOV, focusing on local temporal relationships by presenting image pairs and requiring models to determine their temporal order. Given an image pair where the true temporal order satisfies a certain condition, the model must judge whether the temporal sequence presented by these two images is reasonable, outputting True or False. This task evaluates the model's ability to identify direct temporal relationships and understand subtle temporal transitions between adjacent time points.

\item \textbf{Temporal Position Localization (TPL)} evaluates the model's ability to infer missing temporal positions within a structured sequence. Models are presented with four images: the first three images correspond to positions 1, 3, and 5 in a complete, evenly spaced temporal sequence, while the fourth image represents a missing time point. This task is a simplified version of ISR, where the model does not need to consider all positions for all images, but only needs to assess the matching degree between one image and two possible position orderings.
\end{enumerate}

The TOU task provides comprehensive evaluation of temporal sequence understanding capabilities across multiple difficulty levels and reasoning granularities.
