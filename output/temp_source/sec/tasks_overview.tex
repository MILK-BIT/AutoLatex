TimePerceptBench provides a comprehensive evaluation framework for temporal understanding capabilities in multimodal large language models through two complementary tasks that capture fundamental aspects of temporal reasoning. Our benchmark design leverages remote sensing imagery to provide objective temporal ground truth while systematically assessing model performance across multiple temporal scales and complexity levels. These tasks are specifically designed to isolate different aspects of temporal cognition, enabling fine-grained analysis of model capabilities and limitations.

\textbf{Task Overview}

Drawing inspiration from human temporal perception, we decompose temporal understanding into two core capabilities that reflect different cognitive processes involved in temporal reasoning. Temporal Order Understanding (TOU) evaluates the ability to comprehend chronological sequences and identify temporal relationships within multi-image sequences, representing the directional aspect of temporal cognition. Temporal Interval Understanding (TIU) evaluates the ability to comprehend time spans and identify temporal interval relationships between image pairs, representing the quantitative dimension of temporal understanding---analogous to how vectors possess both direction and magnitude. The multiple subtasks under these two tasks enable comprehensive evaluation while allowing for more systematic analysis of different temporal understanding mechanisms.
