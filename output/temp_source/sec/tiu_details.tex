\textbf{Temporal Interval Understanding (TIU)}

The Temporal Interval Understanding task evaluates the ability to comprehend time spans and identify temporal interval relationships between image pairs. This task addresses the quantitative aspect of temporal reasoning. Our TIU task utilizes the fMoW dataset, which contains over one million images from more than 200 countries with detailed metadata including precise timestamps. We select representative categories including airport\_hangar, airport\_terminal, factory\_or\_powerplant, and other infrastructure types that exhibit temporal evolution patterns. From each category, we extract image sequences from the same geographical locations, generating over 30,000 image pairs spanning different temporal scales, while filtering out image groups potentially affected by cloud cover or other interference based on their metadata. During the sample construction process, to avoid bias in model learning caused by uneven category distribution, we further filter and resample based on change detection results, ensuring that each TIU subtask has 800 groups for training and 200 groups for evaluation.

The TIU task encompasses four complementary subtasks at different granularities and difficulty levels, evaluating distinct aspects of temporal interval understanding:
\begin{enumerate}
\item \textbf{Interval Category Estimation (ICE)} evaluates the model's ability to estimate absolute temporal durations for image pairs. Given two images of the same location taken at two times, the model must classify the time span into one of five categories: A. Days (1--30 days), B. 1--3 Months (31--90 days), C. 3--12 Months (91--365 days), D. 1--2 Years (366--730 days), or E. 2+ Years (731 days or more). This task evaluates the model's ability to perceive temporal magnitude from visual changes and map observed variations to discrete temporal categories.

\item \textbf{Pairwise Interval Comparison (PIC)} focuses on relative temporal interval judgment by presenting two image pairs from the same location at different times. Given four images corresponding to four times, where two images are from the same location and the other two are from another location, the model must determine whether the time interval between the first pair is larger than that of the second pair, outputting True or False. This task evaluates the model's ability to compare temporal intervals without requiring absolute quantification, mitigating the challenge that absolute time is difficult to judge in remote sensing imagery.

\item \textbf{Interval-based Pair Ranking (IPR)} extends PIC by requiring models to rank three image pairs by their temporal spans. Given three groups of image pairs A, B, and C, where each group shows the same location at different times, the model must rank these groups from shortest to longest time span. This task evaluates the model's ability to establish a global temporal ordering across multiple independent span observations, requiring simultaneous comparison and consistent ranking.

\item \textbf{Extremum Interval Identification (EII)} is a simplified version of IPR, assessing the model's ability to identify extreme temporal intervals among multiple options. Given four groups of image pairs---Group A through Group D---where each pair shows the same location at different times, the model only needs to identify which group has the longest time span without ranking all groups by their temporal spans.
\end{enumerate}

The TIU task provides comprehensive evaluation of temporal interval understanding capabilities across absolute and relative temporal reasoning paradigms.
